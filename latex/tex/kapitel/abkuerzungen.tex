% Die längste Abkürzung kann in die eckigen Klammern
% bei \begin{acronym} geschrieben, um einen hässlichen
% Umbruch zu verhindern
%
% ACHTUNG: Sie müssen die Abkürzungen von Hand alphabetisch
%          sortieren. Das passiert nicht automatisch.
\begin{acronym}[IEEE]
% \acro{ABK}{Abkürzung}
% \acro{ACM}{Association of Computing Machinery}
% \acro{PDF}{Portable Document Format}
% \acro{IEEE}{Institute of Electrical and Electronics Engineers}
% \acro{ISO}{International Organization for Standardization}
\acro{NNUE}{Efficiently Updatable Neural Network}
\acro{SIMD}{Single Instruction, Multiple Data}
\acro{SISD}{Single Instruction, Single Data}
\acro{MIMD}{Multiple Instruction, Multiple Data}
\acro{MISD}{Multiple Instruction, Single Data}
\acro{AVX2}{Advanced Vector Extensions 2}
\acro{HCE}{hand-crafted evaluation}
\acro{KNN}{künstliches neuronales Netz}
\acrodefplural{KNN}{künstliche neuronale Netze}
\acro{NN}{neuronales Netz}
\acrodefplural{NN}{neuronale Netze}
\acro{CNN}{Convolutional Neural Network}
\acro{DNN}{Deep Neural Network}
\acro{FNN}{Feedforward Neural Network}
\acro{KIS}{Künstliche Intelligenz für autonome Systeme}
\acro{Lc0}{Leela Chess Zero}
\acro{TCEC}{Top Chess Engine Championship}
\acro{MCTS}{Monte Carlo tree search}
\acro{ReLU}{Rectified Linear Unit}
\end{acronym}
