\chapter{Ergebnisse}
\label{chap:Ergebnisse}

Dieses Kapitel fasst die Ergebnisse dieser Arbeit zusammen. Der Fokus liegt dabei auf der Ermittlung einer Elo-Wertung für die durchgeführten Tests. Die Elo-Zahl, ist eine weit verbreitete Wertungszahl für die Bestimmung der Stärke von Kontrahenten in verschiedenen Sportarten, besonders im Schach. Zuerst wird der Testaufbau geschildert, dass die hier ermittelten Ergebnisse reproduzierbar und nachvollziehbar sind. Außerdem wird die Auswahl der verwendeten Komponenten zum Testen erläutert. Anschließend werden die Tests beschrieben und vorgestellt. Eine Diskussion findet in \autoref{chap:discussion} statt.

\section{Testaufbau}

Schachcomputer sind von Natur aus deterministisch. Deshalb wird zur Vermeidung des immer gleichen Spielablaufs ein Eröffnungsbuch verwendet. Dafür wird das \ac{UHO} V3 von \citeauthor{Pohl2021} \cite{Pohl2021} zusammengestellte Eröffnungsbuch verwendet. \ac{UHO} enthält Eröffnungen aus Spielen starker Schachspieler (2300+ Elo), bei denen laut einer Analyse von Komodo Dragon \cite{KomodoDragon} ein Vorteil für Weiß vorliegt. Diese Eröffnungen eignen sich gut für Schachcomputer, da so weniger Remis gespielt werden als bei ausgeglichenen Eröffnungen. Konkret werden Eröffnungen mit sechs Zügen und einem Vorteil von +0.90 bis +0.99 für Weiß verwendet. Jede Eröffnung wird von beiden Computern mit beiden Farben gespielt.

Der Schachcomputer unterstützt den \ac{UCI}-Standard. Das ermöglicht die Einbindung in gängige Schachprogramme/GUIs und vereinfacht das Ausführen von Self-Play Turnieren und Turnieren gegen andere Schachcomputer. Tests werden mithilfe der cutechess-cli \cite{CutechessRepo} Konsolen-Anwendung durchgeführt. Die Leistung wird anschließend von Ordo \cite{OrdoRepo} berechnet und anhand von Simulationen werden Abweichungen berechnet, der in folgenden Graphen mithilfe von Errorbalken angegeben ist. Die Ordo-Bewertung verhält sich ähnlich wie die Elo-Bewertung, behält jedoch eine höhere Konsistenz zwischen Bewertungen, da alle Ergebnisse gleichzeitig berechnet werden. Im folgenden Text wird die Ordo-Bewertung als Elo referenziert.

Die Spiele werden mit \ac{STC} (10s+0,1s) gespielt, außer anders angegeben. Das heißt beide Seiten haben insgesamt 10 Sekunden und erhalten zusätzlich 0,1 Sekunde pro gespieltem Zug. Für die Spiele wird der von der Hochschule zur Verfügung gestellten Rechner \emph{BigC} verwendet, mit zwei Intel(R) Xeon(R) Gold 6230R CPU @ 2.10GHz. Die Hash Größe des Schachcomputers beträgt ein Gigabyte.

Für die Bewertung der Ergebnisse und Zwischenergebnisse der Testläufe wird Ergebnis Urteilssprengung verwendet. Das heißt, wenn die Bewertung beider Schachcomputer für Anzahl von Zügen einen Wert überschreitet, wird das Spiel beendet und das Ergebnis automatisch erteilt. Konkret gilt, dass ein Computer mit einem Vorteil von mindestens 1000 \ac{CP} für drei Züge automatisch als Sieger gewertet wird. Ein Spiel wird mit remis bewertet, wenn nach 40 Zügen, für acht Züge, die Evaluation nicht mehr als 10 \ac{CP} betrug. Das beschleunigt die Tests deutlich, ohne einen großen Einfluss auf das Ergebnis.

\section{Elo-Entwicklung}

In diesem Kapitel wird die Elo-Entwicklung von der Ausgangsversion mit \ac{HCE} bis hin zu dem Stärksten \ac{NNUE} basierten Version. In \autoref{table:testOverview} sind die durchgeführten Tests aufgelistet. Bei den Tests handelt es sich ausschließlich um Änderungen des Trainings, da er Fokus dieser Arbeit auf dem Training eines \acp{NNUE} liegt. Änderungen an dem Interferenz Code des Schachcomputers sind nicht im Rahmen dieser Arbeit.

\begin{table}[ht]
  \caption{Übersicht alle Trainingsdurchläufe der für diese Arbeit entwickelten Netzwerke. Jeder Test erhält eine ID zur Referenzierung und eine kurze Beschreibung}
  \label{table:testOverview}
  \renewcommand{\arraystretch}{1.2}
  \centering
  \sffamily
  \begin{footnotesize}
    \begin{tabular}{l l}
      \toprule
      \textbf{Test Nr.} & \textbf{Beschreibung}                                                                 \\
      \midrule
      1                 & Erstellung einer Basis Version                                                        \\
      2                 & Mittlere quadratische Fehler-Verlustfunktion statt Kreuzentropie-Verlustfunktion      \\
      3                 & Frisches Training des Netzes mit Simplen $5000$ Knoten Daten                          \\
      4                 & Das Netz aus Test 3 mit der Kombination aller Drei Trainingsdatensätze neu Trainieren \\
      5                 & Das Netz aus Test 3 mit \ac{Lc0} Daten neu Trainieren                                 \\
      6                 & Das Netz aus Test 5 mit der Kombination aller Drei Trainingsdatensätze neu Trainieren \\
      7                 & Frisches Training des Netzes mit Simplen $5000$ Knoten Daten und lambda 0,8           \\
      8                 & Das Netz aus Test 7 mit \ac{Lc0} Daten und lambda 0,8 neu Trainieren                  \\
      \bottomrule
    \end{tabular}
  \end{footnotesize}
  \rmfamily
\end{table}

% match vs marverick 10+0.1 10000 games Maveric (CCRL Rating: 2577)
% Score of NNUE vs MAVERICK: 5265 - 3510 - 1225  [0.588] 10000
% ...      NNUE playing White: 3044 - 1346 - 610  [0.670] 5000
% ...      NNUE playing Black: 2221 - 2164 - 615  [0.506] 5000
% ...      White vs Black: 5208 - 3567 - 1225  [0.582] 10000
% Elo difference: 61.6 +/- 6.5, LOS: 100.0 %, DrawRatio: 12.3 %
% Estimates elo for our engine is 2638,6 +/- 6.5

In einem Trainingslauf wird jede 25 Epochen ein Zwischenergebnis gespeichert. Folgende Graphen zeigen die Elo-Entwicklung der Tests jedes Tests über das Training hinweg. Als Referenzwert wurde anfangs ein Netz mit den in \autoref{chap:trainiung} beschriebenen Traineraufbau trainiert. Das heißt, dass für die Elo-Ermittlung, der in \autoref{table:testOverview} gelisteten Tests, gegen das Referenz-Netz gespielt wurden.

% \begin{figure}
%   \centering
%   \begin{tikzpicture}
%     \begin{axis}[
%         grid=none,
%         axis lines=middle,
%         xlabel={Epoche},
%         ylabel={Elo},
%         xmin=0,
%         xmax=800,
%         x label style={at={(axis description cs:0.5,-0.1)},anchor=north},
%         y label style={at={(axis description cs:-0.15,.5)},rotate=90,anchor=south}
%       ]
%       \addplot plot [error bars/.cd, y dir = both, y explicit] table [blue, x=epoch, y=elo, y error =error, col sep=comma] {src/ordo/version_1_ordo.csv};
%     \end{axis}
%   \end{tikzpicture}
%   \caption{Elo über Epochen.}
%   \label{fig:loss}
% \end{figure}

\autoref{chap:trainer} beschreibt, das die Kreuzentropie-Verlustfunktion verwendet wird. Jedoch wurde in den Grundlagen bereits erläutert, dass es verschiedene Verlustfunktion gibt, welche sich für \ac{NNUE} eignen. Deshalb wurden zwei Tests durchgeführt, welche die Mittlere quadratische Fehler-Verlustfunktion statt der Kreuzentropie verwenden. \autoref{fig:test2_4} zeigt die zwei Tests (Nr. 2 und 3).

\begin{figure}
  \centering
  \begin{tikzpicture}
    \begin{axis}[
        grid=none,
        axis lines=middle,
        xlabel={Epoche},
        ylabel={Elo},
        xmin=0,
        xmax=800,
        x label style={at={(axis description cs:0.5,-0.1)},anchor=north},
        y label style={at={(axis description cs:-0.15,.5)},rotate=90,anchor=south},
        legend pos=south east
      ]
      \addplot plot [error bars/.cd, y dir = both, y explicit] table [x=epoch, y=elo, y error =error, col sep=comma] {src/ordo/version_2_ordo.csv};
    \end{axis}
  \end{tikzpicture}
  \caption{Elo des zweiten Testlaufs. Die Kreuzentropie-Verlustfunktion wurde mit der Mittleren quadratischen Fehler-Verlustfunktion ersetzt}
  \label{fig:test2_4}
\end{figure}

\autoref{chap:inputdata} beschreibt die drei verschiedenen Datensätze die für diese Arbeit verwendet wurden. Unklar ist hingegen in welcher Zusammensetzung und Reihenfolge sie für das Training am besten verwendet werden sollen. Deshalb sind in den folgenden Abbildungen unterschiedliche Zusammensetzungen und Reihenfolgen gezeigt.

Test 3 bildet ein neues Basisnetz, welches die Grundlage für die Tests 4 und 5 bildet. Zum Training des Netzes in Test 3 dient der simpelste der drei verwendeten Datensätze. \autoref{fig:test5_7} zeigt die Stärke von Test 3 und des darauf aufbauenden Tests 5, bei dem alle Trainingsdaten eingesetzt wurden. Test 3 wird in \autoref{fig:test5_8_9} ebenfalls mit Test 5 und 6 verglichen, hier ist zwischen dem Training der simplen Daten und allen Daten ein Trainingslauf mit dem \ac{Lc0} Datensatz.

\begin{figure}
  \centering
  \begin{tikzpicture}
    \begin{axis}[
        grid=none,
        axis lines=middle,
        xlabel={Epoche},
        ylabel={Elo},
        xmin=0,
        xmax=800,
        x label style={at={(axis description cs:0.5,-0.1)},anchor=north},
        y label style={at={(axis description cs:-0.15,.5)},rotate=90,anchor=south},
        legend pos=south east
      ]
      \addplot plot [error bars/.cd, y dir = both, y explicit] table [x=epoch, y=elo, y error =error, col sep=comma] {src/ordo/version_5_ordo.csv};
      \addlegendentry{Test 3}
      \addplot plot [error bars/.cd, y dir = both, y explicit] table [x=epoch, y=elo, y error =error, col sep=comma] {src/ordo/version_7_ordo.csv};
      \addlegendentry{Test 4}
    \end{axis}
  \end{tikzpicture}
  \caption{Elo der Testläufe 3 und 4. Test 3 bildet die Basis für Test 4}
  \label{fig:test5_7}
\end{figure}

\begin{figure}
  \centering
  \begin{tikzpicture}
    \begin{axis}[
        grid=none,
        axis lines=middle,
        xlabel={Epoche},
        ylabel={Elo},
        xmin=0,
        xmax=800,
        x label style={at={(axis description cs:0.5,-0.1)},anchor=north},
        y label style={at={(axis description cs:-0.15,.5)},rotate=90,anchor=south},
        legend pos=south east
      ]
      \addplot plot [error bars/.cd, y dir = both, y explicit] table [x=epoch, y=elo, y error =error, col sep=comma] {src/ordo/version_5_ordo.csv};
      \addlegendentry{Test 3}
      \addplot plot [error bars/.cd, y dir = both, y explicit] table [x=epoch, y=elo, y error =error, col sep=comma] {src/ordo/version_8_ordo.csv};
      \addlegendentry{Test 5}
      \addplot plot [error bars/.cd, y dir = both, y explicit] table [x=epoch, y=elo, y error =error, col sep=comma] {src/ordo/version_9_ordo.csv};
      \addlegendentry{Test 6}
    \end{axis}
  \end{tikzpicture}
  \caption{Elo der Testläufe 3, 5 und 6. Test 3 bildet die Basis für Test 5, welcher wiederum die Basis für Test 6 ist}
  \label{fig:test5_8_9}
\end{figure}

Test 7 und 8 verfolgen einen ähnlichen Ansatz wie die Tests zuvor. Der Unterschied ist, dass bei dem Training und neu Trainieren der Netze, ein gewichtetes arithmetisches Mittel von der Evaluation (in \ac{CP}) und dem Ergebnis des Spiels gebildet wird. In diesem fall, beträgt die Gewichtung 80/20, also bildet die Evaluation 80 \% und das Ergebnis des Spiels 20 \% des erwarteten Ergebnisses. In \autoref{fig:test10_11} sind beide Tests abgebildet.

\begin{figure}
  \centering
  \begin{tikzpicture}
    \begin{axis}[
        grid=none,
        axis lines=middle,
        xlabel={Epoche},
        ylabel={Elo},
        xmin=0,
        xmax=800,
        x label style={at={(axis description cs:0.5,-0.1)},anchor=north},
        y label style={at={(axis description cs:-0.15,.5)},rotate=90,anchor=south},
        legend pos=south east
      ]
      \addplot plot [error bars/.cd, y dir = both, y explicit] table [x=epoch, y=elo, y error =error, col sep=comma] {src/ordo/version_10_ordo.csv};
      \addlegendentry{Test 7}
      % \addplot plot [error bars/.cd, y dir = both, y explicit] table [x=epoch, y=elo, y error =error, col sep=comma] {src/ordo/version_11_ordo.csv};
      % \addlegendentry{Test 8}
    \end{axis}
  \end{tikzpicture}
  \caption{Elo der Testläufe 7 und 8. Test 7 bildet die Basis für Test 8}
  \label{fig:test10_11}
\end{figure}

% overview
% \begin{figure}
%   \centering
%   \begin{tikzpicture}
%     \begin{axis}[
%         grid=none,
%         axis lines=middle,
%         xlabel={Test Nr.},
%         ylabel={Elo},
%         x label style={at={(axis description cs:0.5,-0.1)},anchor=north},
%         y label style={at={(axis description cs:-0.15,.5)},rotate=90,anchor=south},
%         legend pos=south east
%       ]
%       \addplot plot [error bars/.cd, y dir = both, y explicit] table [x=test, y=elo, y error =error, col sep=comma] {src/ordo/overall_ordo.csv};
%     \end{axis}
%   \end{tikzpicture}
%   \caption{Elo aller Tests im Überblick}
%   \label{fig:testOverall}
% \end{figure}

% TODO: elo vegleich von hce und nnue

% TODO: match gegen maveric