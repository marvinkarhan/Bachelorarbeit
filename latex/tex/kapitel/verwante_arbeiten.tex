\chapter{Verwandte Arbeiten}

% AlphaZero erklären:
% eingabe daten, architektur, training
% spiel gegen stockfish 8 -> eingehen auf versuchs aufbau und spezialisierte hardware für AlphaZero (https://www.chessprogramming.org/AlphaZero und http://www.open-chess.org/viewtopic.php?f=5&t=3153)
Schon vor der Entwicklung von \acp{NNUE} gab es einen \ac{NN} basierten Schachcomputer namens AlphaZero, der den schon damals den noch \ac{HCE} basierten Schachcomputer Stockfish vernichtend Schlagen konnte \cite{Silver2017}. AlphaZero wurde \citeyear{Silver2017} von zu Google gehörenden Forschungsunternehmen DeepMind entwickelt. Es erlernte laut DeepMind bereit nach vier Stunden Self-Play reinforcement Training die nötige Spielstärke, um gegen Stockfish zu gewinnen. DeepMinds Ansatz 

Der erstmals durch einen Schachcomputer geschlagene damalige Schachweltmeister \citeauthor{Kasparov2018} \cite{Kasparov2018} gefällt der dynamische und offener Spielstil von AlphaZero, der anders als der von \ac{HCE} basierten Schachcomputer auf konventionellem wissen aufbauende Spielstil. Er beschrieb AlphaZero als Experten und nicht als das Werkzeug eins Experten. Damit deutet \citeauthor{Kasparov2018} darauf hin, dass Schachspieler, besonders Super-Großmeister, diesen Schachcomputer nicht nur für die Analyse ihrer Züge nehmen kann, sondern auch für das Entdecken neuer Spielweisen, die vorher nicht in Betracht gezogen worden wären.

AlphaZero nicht öffentlich zugänglich und wird auch von DeepMinds Seite nicht weiter entwickelt. Andere Entwickler nutzen jedoch die Herangehensweise von AlphaZero und bauen ihr eigens \ac{NN} nach diesem Ansatz. Der aktuell stärkste Nachfolger heißt \ac{Lc0}. \ac{Lc0} hat gemessen an dem letzten \ac{TCEC} Superfinal gegen Stockfish eine Elo von 3586 \cite{haworth202120th} und ist somit wahrscheinlich stärker als AlphaZero. In diesem Kampf gegen Stockfish ging Stockfish mit seiner \ac{NNUE} basierten Variante siegreich hervor. Also bleibt es spannend welcher Ansatz sich durchsetzen wird.