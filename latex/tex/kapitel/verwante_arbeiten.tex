\chapter{Verwandte Arbeiten}

Auch wenn \acp{NNUE} erst seid 2020 gibt, hat es einen großen Einfluss auf die Schachcomputerlandschaft. In der letzten Saison (Saison 22) der \ac{TCEC} \cite{TCEC22} spielen fünf der acht Teilnehmer der höchsten Division mit einer Hybriden \ac{NNUE} Evaluation, die restlichen drei nutzen, einen von AlphaZero etablierten, \ac{NN} Ansatz der später in diesem Kapitel genauer erläutert ist.

% AlphaZero: LCZero, Stoofvlees, ScorpioNN, 
% NNUE: Stockfish, KomodoDragon, Igel, rofChade, SlowChess Blitz 

Stockfish war der erste Schachcomputer mit \ac{NNUE} und manifestierte so, seine Stellung als der stärkste Schachcomputer. Die Entwicklung ist ein Community-Projekt. Das auf SETI@home \cite{SETI2001} basierende Testing Framework Fishtest ermöglicht das Testen tausender Versionen. Alle Änderungen der Codebasis und neue \acp{NNUE} werden durch die Plattform getestet. Stand August 2022 gibt es 289 Entwickler und 1747 Tester die 126 Tausend Test, seit der Entstehung der Plattform 2013, durchgeführt haben \cite{FishtestUsers}. Dieser Ansatz ist ein großer Faktor dafür, wie Stockfish der beste Schachcomputer wurde und auch in zukünftig bleibt. Die meisten anderen \ac{NNUE} Schachcomputer bauen auf der Variante von Stockfish auf.
 
Die Architektur der Stockfish \ac{NNUE} ist aktuell in seiner fünften Version. Es besteht aus einem Feature Set mit 45.056 Eingabeparametern, namens HalfKAv2\_hm, die inkrementell in zwei Farben abhängigen Akkumulatoren aktualisiert werden. Die Ausgabe der Eingabeschicht besteht aus jeweils 520 Ausgabewerten, welche in einen Vektor mit acht Werten und einen mit den restlichen 512 Werten geteilt wird. Die Zwei Vektoren mit acht Werten werden basierend auf der Seite, welche am Zug ist, angepasst. Anhand der Phase des Spiels wird einer der sogenannten Buckets gewählt. Konkret bestimmt die Anzahl der im spiel stehenden Figuren die Spielphase: $\left \lfloor\frac{pieceCount-1}{4}\right \rfloor$. Das gleiche Verfahren wird auch zur Auswahl der Schichten der versteckten Schichten verwendet. Die Buckets beinhalten zwei Schichten, welche die vorher genannten kombiniert 1024 Werte, Gewichten und mit einer Clipped \ac{ReLU} oder der Quadratwurzel einer Clipped \ac{ReLU} aktivieren. Alle Schichten sind Linear und die Quantisierung wir schon während des Trainings angewandt \cite{StockfishNNUE}.

Der Unterschied von HalfKAv2\_hm zu dem in dieser Arbeit verwendeten HalfKP Feature Set, ist das der König selbst als Figur enthalten ist. Jedoch werden die Könige egal welcher Farbe als ein Figurentyp angesehen, da die Belegung ihrer Felder disjunkt ist sie also nie auf demselben Feld stehen können, so werden acht Prozent der Eingabeparameter gepaart. \enquote{hm} steht für \enquote{horizontally mirrored}, zu Deutsch horizontal gespiegelt. Das bedeutet das Brett wird vor der Erstellung der Eingabeparameter gespiegelt, sodass der eigene König immer auf einem der e bis h (je nach Konvention auch a bis d) ränge ist. Das hört sich unintuitiv an und spiegelt nicht die Realität wider, eignet sich trotzdem, da es die Größe des Netzes stark reduziert, aber ein Unterschied in Spielstärke kaum messbar ist \cite{StockfishNNUE}.

Schon vor der Entwicklung von \acp{NNUE} gab es einen \ac{NN} basierten Schachcomputer namens AlphaZero \cite{Silver2017}, der den schon damals den noch \ac{HCE} basierten Schachcomputer Stockfish vernichtend Schlagen konnte. AlphaZero wurde \citeyear{Silver2017} von zu Google gehörenden Forschungsunternehmen DeepMind entwickelt. Es erlernte laut DeepMind bereit nach vier Stunden Self-Play reinforcement Training die nötige Spielstärke, um gegen Stockfish zu gewinnen. DeepMinds nennt den Trainingsansatz \emph{tabula rasa}. Nennenswert ist das zur Suche \ac{MCTS}, statt der normalerweise genutzten Alpha-Beta Suche, verwendet wird. AlphaZero kennt nur die Spielregeln und trainiert sein \ac{CNN} durch Self-Play reinforcement Training. Der Ansatz ist generell anwendbar und hat in den Spielen Schach, Shogi und Go gegen die führenden Computer-Programme gewonnen.

Der erstmals durch einen Schachcomputer geschlagene damalige Schachweltmeister \citeauthor{Kasparov2018} \cite{Kasparov2018} gefällt der dynamische und offener Spielstil von AlphaZero, der anders als der von \ac{HCE} basierten Schachcomputer auf konventionellem wissen aufbauende Spielstil. Er beschrieb AlphaZero als Experten und nicht als das Werkzeug eins Experten. Damit deutet \citeauthor{Kasparov2018} darauf hin, dass Schachspieler, besonders Super-Großmeister, diesen Schachcomputer nicht nur für die Analyse ihrer Züge nehmen kann, sondern auch für das Entdecken neuer Spielweisen, die vorher nicht in Betracht gezogen worden wären.

Leider ist AlphaZero nicht öffentlich zugänglich und wird auch von DeepMinds Seite nicht weiter entwickelt. Andere Entwickler nutzen jedoch die Herangehensweise von AlphaZero und Entwickeln ihr eigens \ac{NN} nach diesem Ansatz. Der aktuell stärkste Nachfolger heißt \ac{Lc0} \cite{Lc0Homepage}. \ac{Lc0} hat gemessen an dem letzten \ac{TCEC} Saison eine Elo von 3586 \cite{TCEC22} und ist somit wahrscheinlich stärker als AlphaZero. \ac{Lc0} belegte in der 22ten \ac{TCEC} Saison den dritten Platz hinter den zwei \ac{NNUE} Schachcomputern Stockfish und KomodoDragon \cite{KomodoDragon}. Es bleibt es spannend welcher Ansatz sich durchsetzen wird.