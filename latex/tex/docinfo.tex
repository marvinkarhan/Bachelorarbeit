% -------------------------------------------------------
% Daten für die Arbeit
% Wenn hier alles korrekt eingetragen wurde, wird das Titelblatt
% automatisch generiert. D.h. die Datei titelblatt.tex muss nicht mehr
% angepasst werden.

% Titel der Arbeit auf Deutsch
\newcommand{\hsmatitelde}{Entwicklung eines \acf{NNUE} zur Evaluation von Schachpositionen}

% Titel der Arbeit auf Englisch
\newcommand{\hsmatitelen}{Development of an \acf{NNUE} for the Evaluation of Chess Positions}

% Weitere Informationen zur Arbeit
\newcommand{\hsmaort}{Mannheim}          % Ort
\newcommand{\hsmaautorvname}{Marvin}        % Vorname(n)
\newcommand{\hsmaautornname}{Karhan} % Nachname(n)
\newcommand{\hsmadatum}{28.09.2022}      % Datum der Abgabe
\newcommand{\hsmajahr}{2022}             % Jahr der Abgabe
\newcommand{\hsmafirma}{} % Firma bei der die Arbeit durchgeführt wurde
\newcommand{\hsmabetreuer}{Prof. Dr. Jörn Fischer, Hochschule Mannheim} % Betreuer an der Hochschule
\newcommand{\hsmazweitkorrektor}{Prof. Dr. Thomas Ihme, Hochschule Mannheim}   % Betreuer im Unternehmen oder Zweitkorrektor
\newcommand{\hsmafakultaet}{I}    % I für Informatik oder E, S, B, D, M, N, W, V
\newcommand{\hsmastudiengang}{IB} % IB IMB UIB CSB IM MTB (weitere siehe titleblatt.tex)

% Zustimmung zur Veröffentlichung
\setboolean{hsmapublizieren}{true}   % Einer Veröffentlichung wird zugestimmt
\setboolean{hsmasperrvermerk}{false} % Die Arbeit hat keinen Sperrvermerk

% -------------------------------------------------------
% Abstract
% Achtung: Wenn Sie im Abstrakt Anführungszeichen verwenden wollen, dann benutzen Sie
%          nicht "` und "', sondern \enquote{}. "` und "' werden nicht richtig
%          erkannt.

% Kurze (maximal halbseitige) Beschreibung, worum es in der Arbeit geht auf Deutsch
% einführung in das Thema - ist architektur, inkrementelle updates
% relevanz und motivation - anwendbar in brettspielen und vl in anderen gebieten
% ziel der Arbeit - Prototyp für ein nnue, (das in einen schacomputer eingebuat wird)
% methode - das netz wird mit merheren milliarden Positionen trainiert
% ergebnisse - nnue schlägt nicht nnue vernichtend
% interpretation
\newcommand{\hsmaabstractde}{Ein \acf{NNUE} ist eine Architektur für neuronale Netzwerke, welche inkrementelle Änderungen des Eingabevektors ausnutzt und effiziente Evaluationen auf CPUs ermöglicht. Sie ist in vielen Brettspielen anwendbar. Diese Arbeit betrachtet das Training eines \acp{NNUE} mit verschiedenen Parametern und die prototypische Implementierung in einem Schachcomputer. Das mit mehreren Milliarden Positionen trainierte Netz schlägt denselben Schachcomputer ohne \ac{NNUE} vernichtend, mit einer Siegesrate von rund 98 \%. Der Schachcomputer mit \ac{NNUE} erreicht eine Spielstärke von 2670 Elo. Die Implementierung eines \acp{NNUE} ist die größte zusammenhängende Verbesserung in der Entwicklung eines Schachcomputers. Sie ist leicht in einen Schachcomputer einzubinden, da es Drop-In Replacement für die herkömmliche Evaluation ist.}

% Kurze (maximal halbseitige) Beschreibung, worum es in der Arbeit geht auf Englisch
\newcommand{\hsmaabstracten}{An \acf{NNUE} is a neural network architecture that exploits incremental changes in the input vector and enables efficient evaluations on CPUs. It is applicable in many board games. This work considers the training of an \ac{NNUE} with various parameters and a prototype implementation in a chess computer. The net is trained with several billion positions and crushes the same chess computer without \ac{NNUE} with a win rate of about 98 \%. The chess computer with \ac{NNUE} reaches a playing strength of 2670 Elo. The implementation of a \ac{NNUE} is the biggest coherent improvement in the development of a chess computer. It is easy to incorporate into a chess computer since it is a drop-in replacement for the conventional evaluation.}
